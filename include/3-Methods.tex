
\chapter{Methods}\label{chap-meth}


This chapter discusses the methods surrounding the research from this thesis. Section \ref{sec:meth-avail} describes the data acquisition and pre-processing steps for the meta-analysis data. Section \ref{meth:features} outlines initial exploration of the data through feature importance. Section \ref{meth:spar} presents the generation of the correlation values and their respective statistical significance values. Section \ref{meth:net_analysis} discusses the network analysis pipeline. Section \ref{meth:viz} explains the visualization methods that were implemented to allow for visual and quantitative comparison of the networks and their structure. Unless otherwise noted, analysis work was performed on a remote Amazon Web Service Elastic Compute Cloud instance running RServer \textbf{cite} and Python \textbf{cite}. Some local analysis was performed using R and Python \textbf{cite}. Network visualization was performed in Cytoscape \citep{Shannon2003_cytoscape}.


%%%%%%%%%%%%%%%%%%%%%%%%%%%%%%%%%%%%%%%%%%%%%%%%%%%%%%%%%%%%%%%%
%%%%%%%%%%%%%%%%%%%%%%%%%%%%%%%%%%%%%%%%%%%%%%%%%%%%%%%%%%%%%%%%
% \section{16S \texorpdfstring{\acrshort{rRNA}}{h} Genomic Data}
\section{Data Acquisition and Pre-processing}\label{sec:meth-avail}
This analysis uses 28 publicly available 16S \acrshort{rRNA} human gut microbiome studies that were gathered and curated by \cite{Duvallet2017} and available with some minor changes to the open-source code made available in her GitHub repository \citep{Duvallet2018}. The respective studies are listed in Table \ref{tab:study_table} along with their associated case disease, and the number of unique samples for the control and case cohorts. As mentioned previously, sequenced 16S \acrshort{rRNA} data can come in various formats depending upon the methods that individual labs use. It is important to consider that community structure from interpreted results differs based upon the V-region targeted in sequencing \citep{Teng2018}. So to avoid study-based artifacts, \citeauthor{Duvallet2017} collapsed the resulting \acrshort{OTU}'s to the genus level. 

\begin{table}[hbtp!]
	\centering
	\begin{tabular}{l l p{0.13\textwidth} l p{0.11\textwidth}}
	\toprule
	   % Dataset ID & Control & \thead{N \\(Controls)} & Case & \thead{N \\ (Cases)}\\
	   Dataset ID & Control & Controls (N) & Case & Cases (N)\\
	   % \hline\hline
	   \midrule
		\cite{edd-singh} & \acrshort{H} & 82 & \acrshort{EDD} & 201 \\
		\cite{cdi-schubert} & \acrshort{H} & 154 & \acrshort{CDI} & 93 \\
		\cite{cdi-schubert} & \acrshort{H} & 154 & \acrshort{nonCDI} & 89 \\
		\cite{cdi-vincent} & \acrshort{H} & 25 & \acrshort{CDI} & 25 \\
		\cite{cdi-youngster} & \acrshort{H} & 4 & \acrshort{CDI} & 19 \\
		\cite{ob-goodrich} & \acrshort{H} & 428 & \acrshort{OB} & 185 \\
		\cite{ob-turnbaugh} & \acrshort{H} & 61 & \acrshort{OB} & 195 \\
		\cite{ob-zupancic} & \acrshort{H} & 96 & \acrshort{OB} & 101 \\
		\cite{ob-ross} & \acrshort{H} & 26 & \acrshort{OB} & 37 \\
		\cite{nash-zhu} & \acrshort{H} & 16 & \acrshort{OB} & 25 \\
		\cite{crc-baxter} & \acrshort{H} & 172 & \acrshort{CRC} & 120 \\
		\cite{crc-zeller} & \acrshort{H} & 75 & \acrshort{CRC} & 41 \\
		\cite{crc-zhaowang} & \acrshort{H} & 54 & \acrshort{CRC} & 44 \\
		\cite{crc-chen} & \acrshort{H} & 22 & \acrshort{CRC} & 21 \\
		\cite{ibd-gevers} & \acrshort{nonIBD} & 16 & \acrshort{CD} & 146 \\
		\cite{ibd-morgan} & \acrshort{H} & 18 & \acrshort{UC}, \acrshort{CD} & 108 \\
		\cite{ibd-papa} & \acrshort{nonIBD} & 24 & \acrshort{UC}, \acrshort{CD} & 66 \\
		\cite{ibd-willing} & \acrshort{H} & 35 & \acrshort{UC}, \acrshort{CD} & 45 \\
		\cite{noguera2016gut} & \acrshort{H} & 34 & \acrshort{HIV} & 205 \\
		\cite{hiv-dinh} & \acrshort{H} & 15 & \acrshort{HIV} & 21 \\
		\cite{lozupone2013alterations} & \acrshort{H} & 13 & \acrshort{HIV} & 23 \\
		\cite{asd-son} & \acrshort{H} & 44 & \acrshort{ASD} & 59 \\
		\cite{asd-kang} & \acrshort{H} & 20 & \acrshort{ASD} & 19 \\
		\cite{t1d-alkanani} & \acrshort{H} & 55 & \acrshort{T1D} & 57 \\
		\cite{t1d-mejia} & \acrshort{H} & 8 & \acrshort{T1D} & 21 \\
		\cite{nash-chan} & \acrshort{H} & 22 & \acrshort{NASH} & 16 \\
		\cite{nash-zhu} & \acrshort{H} & 16 & \acrshort{NASH} & 22 \\
		\cite{art-scher} & \acrshort{H} & 28 & \acrshort{PSA}, \acrshort{RA} & 86 \\
		\cite{mhe-zhang} & \acrshort{H} & 25 & \acrshort{CIRR}, \acrshort{MHE} & 46 \\
		\cite{par-schep} & \acrshort{H} & 74 & \acrshort{PAR} & 74 \\
		\midrule
		\midrule
		Total: & & 1816& & 2210\\
		\bottomrule
    \end{tabular}
    \caption[Table containing information on the respective studies used in this thesis. Also listed are the diseases investigated and the respective control and case cohort sizes.]{Table containing information on the respective studies used in this thesis. Also listed are the diseases investigated and the respective control and case cohort sizes. The respective acronyms are defined as: \acrfull{CRC}, \acrfull{NASH}, \acrfull{PAR}, \acrfull{UC}, \acrfull{CDI}, \acrfull{PSA}, \acrfull{CIRR}, \acrfull{H}, \acrfull{EDD}, \acrfull{CD}, \acrfull{ASD}, \acrfull{T1D}, \acrfull{LIV}, \acrfull{HIV}, \acrfull{IBD}, \acrfull{ART}, \acrfull{OB}, \acrfull{MHE}, \acrfull{RA}, \acrfull{nonCDI}, and \acrfull{nonIBD}.}
    \label{tab:study_table}
\end{table}

\subsection{Acquisition}\label{sec:meth-acq}
The raw study data can be acquired through communication with the various authors or via the \acrfull{NCBI} and \acrfull{ENA} databases and the respective accession number for the study. The data used in this thesis, came directly from \citeauthor{Duvallet2017}'s processing pipeline. 

\citeauthor{Duvallet2017} collected the raw \textit{FASTA} and \textit{FASTQ} files for the studies. This was run through a standardized platform to remove barcodes, and primers, and handle multiplexed files accordingly\footnote{Information on the pipeline is available here: \url{https://amplicon-sequencing-pipeline.readthedocs.io/en/latest/index.html}.}. This pipeline uses clustering at 100\% similarity with USEARCH, and the naive Bayes RDP classifier to assign taxonomy \citep{Wang2007}. The resulting processed raw data has been posted on Zenodo\footnote{This is linked to in \cite{Duvallet2018}, but can be directly accessed here: \url{https://zenodo.org/record/1146764\#.XNBJ1o5KhPY}.}. At this point, the data is now in standard \acrshort{OTU} table format where it contains the counts associated for each genus in each sample. The data was then extracted from the beginning of \citeauthor{Duvallet2017}'s MicrobiomeHD pipeline during the concatenation of all of the study data. To extract the data we ran the beginning of the MicrobiomeHD pipeline, but left out the sample normalization step to retain whole counts in the resulting concatenated dataset. We included the author's collapsing to the genus-level and automatic discarding of unassigned taxa\footnote{We noticed that the unassigned data represented between 0-50\% of the data for a given sample. This represents a significant portion of data and since the taxa annotations were based upon an older database, some portion of this unassigned data could represent newly discovered or un-discovered microbes. The other portion of the data is probably from corrupt or  In an attempt to answer this, we tried to run all of the raw studies through our in-house pipeline. Unfortunately we were not able to finish this effort due to a time constraint limiting our ability to wrangle all of the raw 16s data from the various studies. \textbf{Maybe move this to future work?}}. At this point, we wrote a file containing all \acrshort{OTU} data, and a file containing the concatenated table of all metadata (which includes various information such as class information, sample ID's, \acrshort{NCBI} and \acrshort{ENA} accession numbers, and other information).

\subsection{Pre-processing}\label{sec:meth-processing}
\textbf{\textit{IF I INCLUDE FEATURE SELECTION,MENTION HERE}}

Referencing Table \ref{tab:study_table}, the total number of samples in the data set is  $n_{total}=4026$. Of these, there are $n_{control}=1816$ control samples and $n_{case}=2210$ case samples. At this stage, some of the samples were excluded due to insufficient \acrlong{H} and diseased requirements outlined by \citeauthor{Duvallet2017}. After excluding the samples that should not be considered in the healthy versus diseased sets, we are left with $n_{control}=1751$ and $n_{case}=1973$. In the \acrshort{OTU} data, there are 291 features (genera) containing the counts for the respective taxa in each sample. Some of the analyses and techniques implemented in this study use the data as-is, in its raw count form, and some require additional processing.

\textbf{I think that I will be removing this paragraph.}
To accurately perform feature selection, we normalized the raw counts. In this case we transformed from raw counts to abundance values by normalizing the features in each sample to $[0,1]$. Note that this method is mentioned previously, and may generate biased correlations if used in correlation estimation due to the nature of rare taxa in the microbiota \citep{Friedman2012,McMurdie2014,Gloor2017}. We employ its use because \citeauthor{Duvallet2017} use it to normalize their data for their analyses. 
%Since the raw \acrshort{OTU} counts can vary drastically between the targeted V regions,
 Additionally, the raw counts are passed through a \acrshort{CLR} transformation to compare against.
%Additionally, we take this normalized value and multiply by 40000 and then floor the resulting values so that the data is rarefied and very low count genera are left out. Each method will be used in the feature importance exploration.


Prior to the FastSpar and NetShift analysis we split the data into the healthy and diseased sets and then filter the \acrshort{OTU}s so that each one is present in at least 5\% of the samples in the respective data set. We employ this method to avoid possible division by 0 errors, and to use taxa that are actually present across a large majority of the samples. It also eliminated a potential way for our statistical significance calculations to be biased. In most cases these taxa have either no or a very low variance in their counts, and keeping taxa with less than a 5\% abundance often time limits the number of unique permutations to 1. Essentially, this is another step to ensure artifacts do not influence the end result of the analyses. In total, there were 107 and 111 genera after filtering the features to meet the 5\% threshold in the control (healthy) and case (diseased) set respectively.

\section{Feature Importance}\label{meth:features}
\textbf{Again, this section might be removed.}
Using feature importance techniques allows us to gain an understanding of the data and obtain an overall picture of the structure. As noted in the previous section \textbf{XXXX}, we employ the use of \acrfull{PCA}, random forest feature selection, and \acrfull{LASSO} from the R glmnet package \citep{Friedman2010_glmnet}.

We ran the different feature selection methods on multiple versions of our data in order to obtain benchmarks about genera that may be important members of communities. We looked at the results of feature importance for the combined control and case raw counts data that was normalized via \citeauthor{Duvallet2017}'s fractional abundance method, and the \acrshort{CLR} method. The same analysis for the separated case and control sets. In each case we used the original 291 genera, and the resulting \textbf{XXXX} genera from the 5\% filtering method. Results were then used to compare against key microbes identified through the network analysis section.

\section{Correlation Estimation}\label{meth:spar}
Due to SparCC's unfavorable runtime, memory usage, and high false-positive rate, we used the FastSpar C++ \acrfull{CLI} implementation of SparCC with improved runtime, memory usage, and statistical significance methods \citep{Friedman2012,Watts2018}. FastSpar is run using the raw count data which must be converted to BIOM tsv format\footnote{The description of the BIOM format can be found here: \url{http://biom-format.org/documentation/biom_format.html}} and requires the headers to contain the sample ID and the row names to be the \acrshort{OTU}s. We run the toolkit on our 5\% occurrence threshold data for the combined case and control set, and the respective separated case and control sets. 

After following the initial correlation process, we calculate the exact $p$-value. First, 10000 bootstrap samples were generated and then we inferred correlations for each of the bootstraps in parallel. The resulting correlations were then used to calculate the exact $p$-value for each correlation in the original calculation. 

\textbf{INSERT FIGURE SHOWING PVALUE DISTRIBUTION||}
\textbf{INSERT FIGURE SHOWING CORRELATION DISTRIBUTION||}
\textbf{INSERT FIGURE SHOWING FULLY CONNECTED NETWORKS WITHOUT THRESHOLD FILTERING}

The correlation and statistical significance matrices represent a fully connected network where there exists a correlation connection in the range $[-1,1]$ for each possible \acrshort{OTU} pair. We do not use this fully connected network to compare the control and case networks. In the literature, authors tend to employ some type of threshold value to filter out unnecessary information and connections that may not be significant. This is most frequently done by modifying threshold values to obtain a network that is scale-free or meets some type of average path length or centrality measure, but even this is performed arbitrarily \citep{Perkins2009, Batushansky2016, Romero-Campero2016}. Following arbitrary threshold selection, \citeauthor{Friedman2012}  employ arbitrary thresholds when pruning the resulting networks to analyze. They selected edges greater than 0.3, and omitted unconnected nodes. Arbitrary threshold selection is neither right or wrong, and the prevalence of truly scale-free network structures is a controversial topic in network science today \citep{Broido2019, Barabasi2018, Holme2019}. 

\textbf{REDO THESE VALUES} We tried a combination of threshold measures: keeping all connections with a $p$-value threshold of $p<0.001$; keeping all connections with a $p$-value threshold of $p<0.001$ and correlation value $c$ of $|c| \geq 0.05$; and keeping all connections with a correlation value of $|c| \geq 0.05$. In the networks, if there were nodes that were no longer connected, they were removed for the downstream analysis.

After applying these filtering methods, we became aware that the arbitrary threshold selection might leave out comparable connections shared between the networks. For example, let there be a connection between genera $A \Leftrightarrow B$ in the diseased network with a correlation of 0.25. It will be selected when we filter according to our method. Now let the same connection exist in the healthy network, but with a correlation value of 0.20. The connection in the healthy network will be dropped despite the difference between the two being relatively small. This may be eliminating an important feature in the overlapping sub-graph of the two networks. To remedy this potential artifact generation, we decided to check connection overlaps between the diseased and healthy networks and select an extended correlation value $b$. We first filtered one of the networks by our filtering methods, and then checked all shared connections with the unfiltered network. We take the correlation value $c_f$ that we are filtering the network by and check to see if the unfiltered network shared connection correlation is above the new correlation filter value $c_{f new}$ defined as $c_{f new} = c_f - b$. In the case that the unfiltered network connection correlation value is above $c_{f new}$, we append it to the unfiltered network's filtered results. We repeat the process for the other network so that, after normal filtering and the second filtering are complete, we have two filtered networks that contain the added connections that meet the second correlation filter criteria. After trying many values we implemented a second correlation selection window that we set to 0.1. We will discuss the results from this in the next section. 



\section{Network Analysis}\label{meth:net_analysis}
After converting the correlation and statistical significance matrices to edge lists, we computed the standard graph topology scores using the \textbf{XX library}. This was performed for the combined case and control data, as well as the separate control and case networks. Then, we implemented a version of the \acrshort{NESH} score from \citeauthor{Kuntal2018} applied to the control and case networks. This calculation was performed in \textbf{something}, and the additional topology scores and \acrshort{NESH} were added to a data frame. Since \acrshort{NESH} is direction-agnostic, and only a network topology feature, we do not include correlation values in the calculation. Data was written out to files compatible with our graph visualization software Cytoscape.

\textbf{WHAT AM I MISSING HERE?}

\section{Visualization}\label{meth:viz}
To visualize our data we used the open source platform Cytoscape \citep{Shannon2003_cytoscape}. When writing the graph files, we assigned Hex color codes to the unique genera and families respectively, to aid in differentiating nodes in the network. Family and genera colors were kept the same across the full data set and the control and case sets. In Cytoscape, we set the negative correlation values to blue and positive to red while scaling the size of the edge so that the larger the absolute value of the correlation $|c|$ is the larger the edge is. Driver node sizes identified by the NetShift method are also increased compared to normal nodes in the network. 

We include our own visualization of the networks with our correlation matrices, but we also used the visualizations obtained from the web application provided by \citeauthor{Kuntal2018}\footnote{The web application can be found here: \url{https://web.rniapps.net/netshift/}} for visualizing the network shift. Community shuffle plots from the application are also presented as a tool to better understand the graph visualization of the network shift. The network topology scores and features are listed in tables in the results, and the importance of the metrics will be further discussed there. 

\section{If time permits: ML implementation}





