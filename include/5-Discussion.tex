
\chapter{Discussion}

\bigskip

%%%%%%%%%%%%%%%%%%%%%%%%%%%%%%%%%%%%%%%%%%%%%%%%%%%%%%%%%%%%%%%%
%%%%%%%%%%%%%%%%%%%%%%%%%%%%%%%%%%%%%%%%%%%%%%%%%%%%%%%%%%%%%%%%


\section{Future Work}
 Our methods followed common practices in the literature, but there still remain many opportunities for future work and exploration. 
 
 The careful reader may have noticed that after acquiring and visualizing the filtered correlation networks in Section \ref{sec-res-buff}, we did not utilize the correlation values further. We used the correlation filtering methods to reduce the connectivity in our original correlation matrices, similarly to how the literature tends to employ arbitrary cutoff values to visualize these networks. The resulting networks should contain the structural information that we were curious about. Our time was limited to investigate comparative techniques for similar networks based upon correlations, but we expect that there are relevant methods in the literature.
 
 We decided to employ the NetShift method to identify drivers in the network, but the \citet{Kuntal2018} implementation utilizes undirected and unweighted networks. Further work should focus on modifying the NetShift method or \acrshort{NESH} metric to include correlation value weights. The additional information here could further quantify changes in correlation values and connections between neighbors. This type of comparison would need to consider potential bias arising from the different networks' correlation values potentially not being comparable. 
 
 This study made use of publicly available meta-analysis data, and we split the data into respective healthy and diseased networks. While a general investigation between healthy and diseased gut microbiomes is useful for identifying genera that are drivers in dysbiosis, it would be wise to perform comparisons between healthy guts and individual diseases. We did not perform this type of analysis to identify driving microbes for specific diseases because of the small number of data points for some diseases. We recommend that this comparison be performed for a diseases where there is a sufficiently large sample size. In this case, the methods we followed should be used to build correlation networks for all of the specific diseases. Then the NetShift method could be applied between the healthy correlation network and the respective diseased networks. Results here could shed light on whether disease type (metabolic, cardiovascular, etc.) is associated with certain driving taxa, or whether all diseased guts are similar. 
 
 The development of more robust \acrshort{HT} technologies will be of use for future implementations of these methods. For example, publicly available shallow shotgun sequencing will be easier to combine for meta-analysis work and could allow for this analysis to be performed at the species level. Species level resolution will be vital to researchers aiming to understand the gut microbiome. Additionally, after investigating the \citet{Duvallet2017} data we were unable to build a multiclass classification model to identify specific disease states in the data. Future technologies should be able to give better resolution of microbiome structure, and we expect that with enough samples, the multiclass problem could be solved. 
 
 All of the data in this study comes from a single stool sample from each patient. While we have been able to discern structure in healthy and diseased microbiome networks from this single snapshot, the data does not necessarily represent all of the possible states that these networks could reside in. We know that the gut microbiome has a dynamic community, so temporal data from longitudinal studies would be very useful in understanding both the variability in health states as well as the variability in a given person's microbiome. Such data will contribute to a better understanding of the dynamics in the gut and specifically allow for the identification of driving taxa or key community members.

Identifying structure and key drivers in these microbial networks is just a starting point. Once found it will be useful to begin modeling actual microbe-microbe metabolic interactions.  \citet{Magnusdottir2016} are currently reconstructing the metabolic pathways for microbes in the gut. With their reconstructions it is possible to model the dynamics of these communities through flux balance and co-growth simulations to predict how communities change over time. It is also useful for identifying microbe-microbe relationships, and the \citeauthor{Magnusdottir2016} work is already being used for such simulations\footnote{\url{https://opencobra.github.io/cobratoolbox/stable/tutorials/tutorialMicrobeMicrobeInteractions.html} }.  An extension of this will be to investigate how metabolites produced in the gut impact the human host, and how metabolites from the human host will impact the gut. This type of research is going to be useful for the development of gut-based microbiome therapies and will expedite the research and development process in the wet-lab. 

% While applicable to the human gut microbiome, could use this in any scenario containing different groups with compositional data.
%%%%%%%%%%%%%%%%%%%%%%%%%%%%%%%%%%%%%%%%%%%%%%%%%%%%%%%%%%%%%%%%
%%%%%%%%%%%%%%%%%%%%%%%%%%%%%%%%%%%%%%%%%%%%%%%%%%%%%%%%%%%%%%%%
\section{Societal and Ethical Aspects}
Human microbiome research is very reliant on real patient data. Because of this reliance there are many regulations in place to ensure the protection of the identity of patients and their medical history. Such studies require the patient to indicate the scope of their consent for data usage. Some patients may opt for their data to only be used in a specific study, while others are open to their data being used across multiple types of studies. It is up to the life sciences, healthcare, and academic industries to follow these regulations and patient consent to avoid misuse of data.

If data is not anonymized, patients could unwillingly have their personal information and medical history leaked to the public. Such a blatant violation of individual's privacy is often times against the law, and could negatively impact a patient's future. Research organizations go through many steps to ensure that data is only accessed within the scope of consent. Often times private patient data is stored in databases that log all accessions from researchers in the organization. This is done to track the usage of data and to identify individuals who may use the data maliciously.

The data in this thesis comes from an aggregation of 28 studies which were all followed standard medical regulations of clinical data. While there was associated metadata available for each gut microbiome sample, the different labs anonymized their patient information and followed the scope of their patients' consent. Further research should follow the regulations put in place by governing health bodies, and researcher should be sure to be aware of the societal and personal impacts of misuse of human gut microbiome data.

%%%%%%%%%%%%%%%%%%%%%%%%%%%%%%%%%%%%%%%%%%%%%%%%%%%%%%%%%%%%%%%%
%%%%%%%%%%%%%%%%%%%%%%%%%%%%%%%%%%%%%%%%%%%%%%%%%%%%%%%%%%%%%%%%
\section{Conclusion}

 In this thesis we used different statistical and graph theoretic methods to identify nodes that are closely associated with overall structure change between healthy and diseased networks. After filtering our raw \acrshort{OTU} data so that taxa occured in at least 5\% of our samples, we estimated fractional abundances for the taxa in each sample by drawing from a Dirichilet distribution. The resulting correlation networks generated from the fractional abundances were then filtered under the assumption that the network in the gut is sparse. From here we implemented an extended filtering threshold in order to reduce artifacts that could be introduced by a basic cutoff value. Then the resulting networks were compared using the \acrshort{NetShift} methodology and we identified 9 ``driving'' taxa that contributed to the re-wiring of a healthy to diseased gut microbiome.

These findings highlight the structural diversity of real-world networks and the need for new theoretical explanations of these non-scale-free patterns \citep{Broido2019}. In this thesis, systems and computational biology play a role in understanding the behavior of the gut microbiota. However, current knowledge of the human gut microbiome is still in its infancy. There still exists a large need for further research into microbial interactions and behaviors in the gut. Development of \acrshort{HT} technologies will aid many aspects of microbiome research. Particularly relevant influences will be on metagenomic resolution as well as further research into metabolic pathways, microbe-microbe interactions, and microbe-host interactions. Together this research should uncover mechanisms of action relating to community stability in the gut microbiome and will potentially allow for the development of treatments for a multitude of diseases.

