
\chapter{Discussion}

\bigskip

%%%%%%%%%%%%%%%%%%%%%%%%%%%%%%%%%%%%%%%%%%%%%%%%%%%%%%%%%%%%%%%%
%%%%%%%%%%%%%%%%%%%%%%%%%%%%%%%%%%%%%%%%%%%%%%%%%%%%%%%%%%%%%%%%
\section{Future Work}


%%%%%%%%%%%%%%%%%%%%%%%%%%%%%%%%%%%%%%%%%%%%%%%%%%%%%%%%%%%%%%%%
%%%%%%%%%%%%%%%%%%%%%%%%%%%%%%%%%%%%%%%%%%%%%%%%%%%%%%%%%%%%%%%%
\section{Societal and Ethical Aspects}
% The reliance of big data cancer statistics on large amounts of detailed patient data brings with it both opportunities and risks. While the availability and analyzability of a continuous stream of abundant biological and health-related data for distinct individuals will bring about great advances in personalized treatment, a great emphasis has to be placed on data protection and ethical use of personal data. 
% For instance, making patient data related to behavioral risk factors (such as smoking or an unhealthy diet) available to health insurers, has the potential to improve public health by encouraging healthy behavior through flexible premiums. 
% If such data can be related to preexisting conditions, however, the availability of well-interpretable personalized data sets puts people with genetic predispositions for certain diseases at an unfair disadvantage. This means that healthcare professionals and researchers have a moral obligation to ensure proper data anonymization and protection at each step of cancer research. Results for certain patient groups should always be put into perspective with respect to their societal and personal impacts.

%%%%%%%%%%%%%%%%%%%%%%%%%%%%%%%%%%%%%%%%%%%%%%%%%%%%%%%%%%%%%%%%
%%%%%%%%%%%%%%%%%%%%%%%%%%%%%%%%%%%%%%%%%%%%%%%%%%%%%%%%%%%%%%%%
\section{Conclusion}

These findings highlight the structural diversity of real-world networks and the need for new theoretical explanations of these non-scale-free patterns \citep{Broido2019}.