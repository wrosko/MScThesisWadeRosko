
\textbf{Biological Correlation Network Analysis Derived from Compositional Data}\\
with Applications to the Human Gut Microbiota\\
WADE ROSKO\\
Department of Mathematical Sciences\\
Chalmers University of Technology \setlength{\parskip}{0.5cm}

\thispagestyle{plain}			% Suppress header 
\setlength{\parskip}{0pt plus 1.0pt}
\section*{Abstract}
% Abstract here
Over the last decade, the human gut microbiota has been identified as a contributing factor to human health. As a result there have been advances in high-throughput technologies that have allowed for increasing amounts of information on the microbiota to be acquired. These technologies have opened the door to a burgeoning field of research in human biology that requires an interdisciplinary approach to better understand the complex relationships within microbial systems and potential interactions with a human host's physiology. One of the most abundant types of data used in the field comes from metagenomics, which is compositional in nature and presents a challenge for identifying key microbes in a microbial community. This study focuses on creating comparable correlation networks derived from healthy and diseased human gut microbial count data. We utilize techniques from compositional data correlation generation methods to create control and case microbial networks. We then apply filtering methods to identify important network connections in the respective systems. From here, we compare the networks using overlapping sub-graphs in order to identify microbes that may influence dysbiosis in the gut microbiota. Finally, we generate visual representations of the networks.

% KEYWORDS (MAXIMUM 10 WORDS)
\vfill
Keywords: systems biology, human gut microbiome, correlation networks, community detection, complex networks, compositional data, metagenomics

\newpage				% Create empty back of side
\thispagestyle{empty}
\mbox{}